% Created 2018-01-09 Tue 01:03
% Intended LaTeX compiler: pdflatex
\documentclass[a4paper,oneside]{article}
\usepackage[utf8]{inputenc}
\usepackage[T1]{fontenc}
\usepackage{graphicx}
\usepackage{grffile}
\usepackage{longtable}
\usepackage{wrapfig}
\usepackage{rotating}
\usepackage[normalem]{ulem}
\usepackage{amsmath}
\usepackage{textcomp}
\usepackage{amssymb}
\usepackage{capt-of}
\usepackage{hyperref}
\hypersetup{colorlinks=true,linkcolor=blue}
\setcounter{secnumdepth}{2}
\author{Santhanaraj Chellaiah}
\date{Jan 8, 2018}
\title{Cycle Time Reduction of D-301 Blender}
\hypersetup{
 pdfauthor={Santhanaraj Chellaiah},
 pdftitle={Cycle Time Reduction of D-301 Blender},
 pdfkeywords={Cycle time, process optimization, instrumentation, controller, NC-701, unit based control},
 pdfsubject={},
 pdfcreator={Emacs 27.0.50 (Org mode 9.1.5)}, 
 pdflang={English}}
\begin{document}

\maketitle
\section{Abstract}
\label{sec:orgc2ee538}
The D-301 blender is the one that blends NC-701 (highly viscous
liquid) with different products to produce various required
compositions. NC-701 is unloaded from a truck to the blender manually
and the blending process is intiated. This however has limited the
availability of blender to a point where the blender can be used only
if the truck is available, thus increasing the cycle time for an
output from D-301. This paper proposes the introduction of a buffer
storage tank D-12516 along with piping, valves, pumps and other
instrumentation to reduce the cycle time of the D-301 blender.

\section{Keywords}
\label{sec:orgbb1259a}
\emph{Cycle time, process optimization, instrumentation, controller, NC-701, unit based control}

\section{Introduction}
\label{sec:org44e41e8}
It is proposed that the material (NC-701) from the truck be unloaded
to a buffer storage tank D-12516. D-12516 will serve as a holding tank
when the trucks are available to unload. The material can be
transferred from D-12516 to D-301 as and when required. Material
unloading from the truck will be effected by pump MP-516T. This pump
could either transfer the material to D-12516 or to D-301
directly. The material stored in D-12516 can be transferred to D-301
using MP-516A pump.  

The following are the two benefits that are expected out of this proposal: 
\begin{itemize}
\item The cycle time of D-301 will be greatly reduced
\item The truck waiting times can be drastically reduced resulting in lower cost
\item The product stored in D-12516 can be stored and maintained at a
temperature, most suited for blending
\end{itemize}
\section{Overall Scheme}
\label{sec:org01ee439}
Below is the overall scheme for the project
\{r\}\{0.4\textwidth\} \url{file:///home/prasanna/Documents/git/CSR/Overall\_Scheme}
\section{Process Narrative}
\label{sec:org4ba112a}
The addition of a buffer storage tank calls for a operating procedure
to unload from the truck, recycle the product (and regulate
temperature, if required) and transfer to D-301 when required. It
should also specify when a direct transfer from tank truck can be
initiated toward the blender D-301. The following section details the
process narrative using state based (or unit based control).
\subsection{State based control}
\label{sec:org6e6a2a7}
The following is the abstract from the white paper presented by David
A.Huffman on benefits of State Based Control.
\begin{quote}
State Based Control is a plant automation control design based on the
principle that all process facilities operate in recognized, definable
Process States that represent a variety of normal and abnormal
conditions of the process. State Based Control, implemented with the
latest developments in object-based technologies, delivers direct
benefits to its adopters in a variety of Operational Excellence
categories. It results in productivity increases, higher asset
utilization of both people and process, automated responses and
recovery for abnormal conditions and provides an environment for
knowledge capture directly into the control design.
\end{quote}
The proposal intends leverage the State Based Control scheme for this
implementation. The various steps involved in control and operation of
the plant are listed below: 
\begin{enumerate}
\item Maintenance Wait
\item Process Wait
\item Recycle
\item Transfer
\item Unload
\item Direct Transfer
\item Line Clear
\end{enumerate}
\subsection{Steps}
\label{sec:org5231187}
\subsubsection*{Maintenance Wait}
\label{sec:orga2e71ce}
In this step, the instruments and the control system are in
maintenance. All the process valves will assume thier safe state. The
outputs from the control sytem will be in Fail-safe state. All (or
most) of the instruments are in de-energized state. Most of (or all
non-critical) the alarms are disabled in this step.
\subsubsection*{Process Wait}
\label{sec:org3124092}
This is the step in which all the instruments, valves and the control
system remain energized. The plant is preparing for start up. The
maintenance/operation team should ensure that all (or almost all)
instruments and valves are in healthy condition. Most of the alarms
are enabled in this step. Maintenance/operation would look of
instrument out of service alarms. If any of the devices are in
"out-of-service" state, they should be fixed and put in service for a
effect a start up.
\subsubsection*{Recycle}
\label{sec:org800d4db}
NC-701 is a highly viscous liquid which needs to be agitated
periodically to maintain consistency of the product. During the
recycle step, the temperature of the product can be controlled, if
required. NC-701 is ideal for blending when its temperature is around
45 degree Celcius. 

During the winter months, the product temperature drops because of
ambient tempertuare. In order to compensate and supplement heat to the
product, NC-701 can be passed through a heat exchanger in order to
raise its temperature. The NC-701 would pass through the tube side of
the exchanger. The shell side would pass a 30 barg steam from the
plants' existing steam header. The flow rate of NC-701 through the
heat exchanger E-12516H can be effected by throttling the control
valve CV-20813. If the product temperature drops, the steam control
valve CV-20814 should approach 100\% open position while the CV-20813
should approach around 50\% of opening. The best ratio of opening the
steam line vs opening the NC-701 line should be ascertained during
operation / trail run for utmost efficiency

During the summer months, the product temperature could go way beyond
the desired setpoint and might require cooling before being
blended. For this case, we can close CV-20813 and also the steam valve
CV-20814 and let the NC-701 to recycle through CV-20812. The tube side
of the exchanger E-12516C would pass NC-701. The shell side would pass
the cooling tower water to effect cooling of NC-701. The flow rate of
NC-701 is adjusted using the control valve CV-20812 to effect the
desired cooling. Note that, the cooling tower water does not have any
controls on it which is unlike the steam control line which is used in
the heating circuit.
\subsubsection*{Transfer}
\label{sec:org2146711}
During this step, the material from D-12516 will be transferred to
D-301 blender. The pump MP-516A will be used to effect this transfer
via the ON-OFF valves EV-20811, ABV-20815 and ABV-20850. During the
transfer step, it should also be ensured that the control valves
CV-20812 and CV-20813 remain completely closed to avoid any
recirculation. The nitrogen purge line should also remain isolated (or
closed) using EV-20816.  Before a transfer is initiated, sufficient
level in D-12516 must be ensured. This will prevent the dry running of
MP-516A which might in turn damage the seal of the pump. D-12516
should be at least at 60\% level (arbitrary value) to intiate a
transfer.
\subsubsection*{Unload}
\label{sec:org2cb4b1f}
\subsubsection*{Direct Transfer}
\label{sec:org8cced98}
\subsubsection*{Line Clear}
\label{sec:org181f0e9}

\section{Control Narrative}
\label{sec:org12d6324}
\section{Bill of Material}
\label{sec:orgab8dd84}
\section{Conclusion}
\label{sec:org47beee0}
\section{References}
\label{sec:orgbc17407}
\href{https://www.controlglobal.com/assets/knowledge\_centers/abb/assets/Benefits-of-state-based-control-white-paper.pdf}{1. Benefits of State Based Control}
\begin{enumerate}
\item 
\end{enumerate}
\end{document}

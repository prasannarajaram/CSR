% Created 2018-02-28 Wed 13:43
% Intended LaTeX compiler: pdflatex
\documentclass[a4paper,oneside]{article}
\usepackage[utf8]{inputenc}
\usepackage[T1]{fontenc}
\usepackage{graphicx}
\usepackage{grffile}
\usepackage{longtable}
\usepackage{wrapfig}
\usepackage{rotating}
\usepackage[normalem]{ulem}
\usepackage{amsmath}
\usepackage{textcomp}
\usepackage{amssymb}
\usepackage{capt-of}
\usepackage{hyperref}
\hypersetup{colorlinks=true,linkcolor=blue}
\setcounter{secnumdepth}{4}
\author{Santhanaraj Chellaiah}
\date{Jan 8, 2018}
\title{Cycle Time Reduction of D-301 Blender}
\hypersetup{
 pdfauthor={Santhanaraj Chellaiah},
 pdftitle={Cycle Time Reduction of D-301 Blender},
 pdfkeywords={Cycle time, process optimization, instrumentation, controller, NC-701, unit based control},
 pdfsubject={},
 pdfcreator={Emacs 27.0.50 (Org mode 9.1.6)}, 
 pdflang={English}}
\begin{document}

\maketitle
\begin{abstract}
The D-301 blender is the one that blends NC-701 with different
products to produce various required compositions. NC-701 is unloaded
from a truck to the blender manually and the blending process is
initiated. This however has limited the availability of blender to a
point where the blender can be used only if the truck is available,
thus increasing the cycle time for an output from D-301. This paper
proposes the introduction of a buffer storage tank D-12516 along with
piping, valves, pumps and other instrumentation to reduce the cycle
time of the D-301 blender.
\end{abstract}

\section{Keywords}
\label{sec:org13e1735}
\emph{Cycle time, process optimization, instrumentation, controller, NC-701, unit based control}
\clearpage
\clearpage
\section{Introduction}
\label{sec:org0f9ee60}
It is proposed that the material (NC-701) from the truck be unloaded
to a buffer storage tank D-12516. D-12516 will serve as a holding tank
when the trucks are available to unload. The material can be
transferred from D-12516 to D-301 as and when required. Material
unloading from the truck will be effected by pump MP-516T. This pump
could either transfer the material to D-12516 or to D-301
directly. The material stored in D-12516 can be transferred to D-301
using MP-516A pump.  

The following are the two benefits that are expected out of this proposal: 
\begin{itemize}
\item The cycle time of D-301 will be greatly reduced
\item The truck waiting times can be drastically reduced resulting in lower cost
\item The product stored in D-12516 can be stored and maintained at a
temperature, most suited for blending
\end{itemize}
\clearpage
\section{Overall Scheme}
\label{sec:orgcede2cd}
Below is the overall scheme for the project. The tag numbers for the
various equipment shown in overall scheme are indicative. Tags and IO
type to be finalized based on Process and Control narrative
finalization.
\begin{figure}[htbp]
\centering
\includegraphics[width=15cm,height=12cm]{/home/prasanna/Documents/git/CSR/images/Overall_Scheme.png}
\caption{Schematic showing truck unloading, D-12516 tank and path to D-301}
\end{figure}
\clearpage
\section{Process Narrative}
\label{sec:orge9875e9}
The addition of a buffer storage tank calls for a operating procedure
to unload from the truck, recycle the product (and regulate
temperature, if required) and transfer to D-301 when required. It
should also specify when a direct transfer from tank truck can be
initiated toward the blender D-301. The following section details the
process narrative using state based (or unit based control).
\subsection{State based control}
\label{sec:orgbc87b5b}
The following is the abstract from the white paper presented by David
A.Huffman on benefits of State Based Control.
\begin{quote}
State Based Control is a plant automation control design based on the
principle that all process facilities operate in recognized, definable
Process States that represent a variety of normal and abnormal
conditions of the process. State Based Control, implemented with the
latest developments in object-based technologies, delivers direct
benefits to its adopters in a variety of Operational Excellence
categories. It results in productivity increases, higher asset
utilization of both people and process, automated responses and
recovery for abnormal conditions and provides an environment for
knowledge capture directly into the control design.
\end{quote}
The proposal intends leverage the State Based Control scheme for this
implementation. The various steps involved in control and operation of
the plant are listed below: 
\begin{enumerate}
\item Maintenance Wait
\item Process Wait
\item Recycle
\item Transfer
\item Unload
\item Direct Transfer
\item Line Clear
\end{enumerate}
\subsection{Steps}
\label{sec:orgc1339b2}
\subsubsection{Maintenance Wait}
\label{sec:org3aadfb3}
In this step, the instruments and the control system are in
maintenance. 

All the process valves will assume their safe state. The
outputs from the control system will be in Fail-safe state. All (or
most) of the instruments are in de-energized state. Most of (or all
non-critical) the alarms are disabled in this step.
\subsubsection{Process Wait}
\label{sec:orgf567de9}
This is the step in which all the instruments, valves and the control
system remain energized. The plant is preparing for start up. 

The maintenance/operation team should ensure that all (or almost all)
instruments and valves are in healthy condition. Most of the alarms
are enabled in this step. Maintenance/operation would look of
instrument out of service alarms. If any of the devices are in
"out-of-service" state, they should be fixed and put in service for a
effect a start up.
\subsubsection{Recycle}
\label{sec:orgaa4ce4e}
During the recycle step, the temperature of the product can be
controlled, if required. NC-701 is ideal for blending when its
temperature is around 45 degree Celsius. NC-701 is a highly viscous
liquid which needs to be agitated periodically to maintain consistency
of the product.

During the winter months, the product temperature drops because of
ambient temperature. In order to compensate and supplement heat to the
product, NC-701 can be passed through a heat exchanger in order to
raise its temperature. The NC-701 would pass through the tube side of
the exchanger. The shell side would pass a 30 barg steam from the
plants' existing steam header. The flow rate of NC-701 through the
heat exchanger E-12516H can be effected by throttling the control
valve CV-20813. If the product temperature drops, the steam control
valve CV-20814 should approach 100\% open position while the CV-20813
should approach around 50\% of opening. The best ratio of opening the
steam line vs opening the NC-701 line should be ascertained during
operation / trail run for utmost efficiency

During the summer months, the product temperature could go way beyond
the desired set-point and might require cooling before being
blended. For this case, we can close CV-20813 and also the steam valve
CV-20814 and let the NC-701 to recycle through CV-20812. The tube side
of the exchanger E-12516C would pass NC-701. The shell side would pass
the cooling tower water to effect cooling of NC-701. The flow rate of
NC-701 is adjusted using the control valve CV-20812 to effect the
desired cooling. Note that, the cooling tower water does not have any
controls on it which is unlike the steam control line which is used in
the heating circuit.
\subsubsection{Transfer}
\label{sec:orgaad1953}
During this step, the material from D-12516 will be transferred to
D-301 blender. 

The pump MP-516A will be used to effect this transfer via the ON-OFF
valves EV-20811, ABV-20815 and ABV-20850. During the transfer step, it
should also be ensured that the control valves CV-20812 and CV-20813
remain completely closed to avoid any re-circulation. The nitrogen
purge line should also remain isolated (or closed) using EV-20816.
Before a transfer is initiated, sufficient level in D-12516 must be
ensured. This will prevent the dry running of MP-516A which might in
turn damage the seal of the pump. D-12516 should be at least at 60\%
level (arbitrary value) to initiate a transfer.
\subsubsection{Unload}
\label{sec:orgc38f0e2}
During the unload step, NC-701 is unloaded from the truck to
D-12516. 

The pump MP-516T is utilized to pump the material from the truck to
the tank. Since unloading requires manual connection of the unload
hose to the tank truck, a signal to indicate ready-to-unload is
provided by the operator. This signal will serve as a permissive to
starting MP-516T pump. The material will flow via the pump MP-516T,
ABV-20815 and CV-20813. Note that CV-20813 valve leads into the
E-12516H heat exchanger. During unload, the steam supply to the
E-12516H is shut off. The control valve CV-20813 is a fail-open
valve. It should be ensured that EV-20811 and EV-20816 remain closed
during the unload step.
\subsubsection{Direct Transfer}
\label{sec:orgefab905}
During the Direct transfer step, the material from the truck is
directly transferred to the D-301 blender. This is the present set
up. Hence this functionality is being retained.

In order to effect a direct transfer, a signal from D-301 blender
control system is required to inform that the material can be accepted
by the D-301 blender. After receipt of this signal, the operator shall
connect the hose to the tank truck and start the MP-516T pump for the
direct transfer to happen. The NC-701 will flow through EV-20851 and
EV-20850 to reach D-301. It should be ensured to close EV-20816 and
ABV-20815 during the direct transfer step to avoid material flowing
into the D-12516 circuit or into the N\(_{\text{2}}\) line.
\subsubsection{Line Clear}
\label{sec:org18a299e}
The purpose of line clear step is to ensure that NC-701 does not clog
the line leading to the blender D-301.

The N\(_{\text{2}}\) is purged by opening EV-20816 and EV-20850 valves. N\(_{\text{2}}\)
enters the D-301 carrying with it any residual NC-701 and leaves D-301
through a N\(_{\text{2}}\) vent. Line clear operation should be carried out for a
minimum of 30 minutes (arbitrary value) to ensure complete purging of
the line leading to D-301. During this step, it should be ensured to
close the EV-20851 and ABV-20815 to prevent N\(_{\text{2}}\) from flowing
elsewhere than desired.
\clearpage
\section{Step Transition}
\label{sec:orgf4c9774}
During start-up, the process would start in Maintenance wait step. The
transition (or changeover) from one step to another (like from
maintenance wait to process wait step) may be initiated either
manually or may happen automatically based on process conditions. 
\begin{verbatim}
C                       +-----------------+                  
C                       |   MAINTENANCE   |                  
C                       |      WAIT       |                  
C                       +-----------------+                  
C                        M |           ^                     
C                          V           | M                   
C                       +-----------------+                  
C         +-------------|     PROCESS     |<-----------+    
C         |           M |      WAIT       |            |     
C         |             +--+--------------+            |     
C         |              M |            ^              |     
C         |                V            | M/A          |     
C         |             +---------------+-+            |     
C         |             |     RECYCLE     |<-----------+     
C         |       +-----|                 |<------+    |     
C         |       |   M +--------+--------+       |    |     
C         |       |            A |                |    |     
C         |       |              V                |    |     
C         |       |     +-----------------+  A    |    |     
C         |       |     |     TRANSFER    |---+   |    |     
C         |       |     |                 |   |   |    |     
C         |       |     +-----------------+   |   |    |     
C         |       |                           |   |    |     
C         |       |                           |   |    |     
C         |       |     +-----------------+   |   |    |     
C         |       +---->|     UNLOAD      |-- | --+    |     
C         +------------>|                 | M |        |     
C         |       |     +-----------------+   |        |     
C         |       |                           |        |     
C         |       |                           |        |     
C         |       |     +-----------------+   |        |     
C         +------------>|     DIRECT      |   |        |     
C         |       |     |    TRANSFER     |   |        |     
C         |       |     +--------+--------+   |        |     
C         |       |            M |            |        |     
C         |       |              V            |        |     
C         |       |     +-----------------+   |        |     
C         |       +---->|    LINE CLEAR   |<--+  A     |     
C         +------------>|                 |------------+    
C                       +-----------------+                  
	M denotes a manual transition initiated by the operator
	A denotes an automatic transition based on process condition
\end{verbatim}

\clearpage
\section{Control Narrative}
\label{sec:org75faa61}
The control narrative will pictorially represent the process
conditions that are required in each of the steps. There are no valve
line up and vessel line up requirements for Maintenance Wait and
Process Wait steps as there are no actions/functions.  As a general
convention, the equipment shown in green colour are
open/running. Similarly, equipment's in red colour are either
closed/stopped.
\subsection{Recycle}
\label{sec:orgdb748f8}
Recycle step has the possibility of either heating or cooling the
material. When the yellow colored valves are in open position heating
shall take place. Cooling of material shall happen when the blue
coloured valves are open.
\begin{figure}[htbp]
\centering
\includegraphics[width=15cm,height=12cm]{/home/prasanna/Documents/git/CSR/images/Recycle.png}
\caption{Recycle step}
\end{figure}
\clearpage
\subsection{Transfer}
\label{sec:org7e08f27}
During transfer step, in addition to the lining up of the transfer
valves, the valves in the recirculation line of the pump (CV-20813 or
CV-20812) shall be kept open to allow for minimum re-circulation. The
purpose of re-circulation is to prevent the pump from deadheading in
case if the blender D-301 circuit trips and closes blender inlet
valves
\begin{figure}[htbp]
\centering
\includegraphics[width=15cm,height=12cm]{/home/prasanna/Documents/git/CSR/images/Transfer.png}
\caption{Transfer from D-12516 to D-301}
\end{figure}
\clearpage
\subsection{Unload}
\label{sec:orge301371}
\begin{figure}[htbp]
\centering
\includegraphics[width=15cm,height=12cm]{/home/prasanna/Documents/git/CSR/images/Unload.png}
\caption{Unload from truck to D-12516}
\end{figure}
\clearpage
\subsection{Direct Transfer}
\label{sec:orgcfabd35}
\begin{figure}[htbp]
\centering
\includegraphics[width=15cm,height=12cm]{/home/prasanna/Documents/git/CSR/images/DirectTransfer.png}
\caption{Unload from truck and Direct transfer to D-301}
\end{figure}
\clearpage
\subsection{Line Clear}
\label{sec:org773a053}
\begin{figure}[htbp]
\centering
\includegraphics[width=15cm,height=12cm]{/home/prasanna/Documents/git/CSR/images/LineClear.png}
\caption{Line clear}
\end{figure}
\clearpage
\section{P \& ID}
\label{sec:org3a1aeff}
<< \emph{Insert P \& ID here} >>
\clearpage
\section{IO list}
\label{sec:org4745326}
\begin{center}
\begin{tabular}{lllllll}
Tag & IO \# & I/O & Description with & Equip. & Min. & Max.\\
Name &  & Type & Engg.units & Name & Value & Value\\
\hline
PIT20804 & 181 & AI & N2 Pad Pressure [inH2O] & D12516 & 0 & 25\\
TT20806 & 182 & AI & Temperature [degC] & D12516 & 0 & 200\\
TT20807 & 162 & AI & Recycle line temperature [degC] & D12516 & 0 & 200\\
JT20809 & 180 & AI & Power meter [hp] & MP-516A & 0 & 25\\
TT20808 & 165 & AI & Suction temperature [degC] & MP-516A & 0 & 200\\
PIT20810 & 163 & AI & Discharge pressure [psig] & MP-516A & 0 & 200\\
TT20853 & 159 & AI & Suction temperature [degC] & MP-516T & 0 & 200\\
PIT20852 & 135 & AI & Discharge pressure [psig] & MP-516T & 0 & 200\\
JT20854 & 432 & AI & power meter [hp] & MP-516T & 0 & 25\\
EY20854 & 401 & DO & Pump should run F.Off & MP-516T & Off & On\\
EY20809 & 130 & DO & Pump should run F.Off & MP-516A & Off & On\\
EV20850 & 112 & DO & TT to D-301 BV & D12516 & Close & Open\\
EV20851 & 403 & DO & Discharge BV F.C. & D12516 & Close & Open\\
EV20816 & 131 & DO & Outlet N2 clear BV F.C & D12516 & Close & Open\\
EV20815 & 133 & DO & Outlet BV F.C & D12516 & Close & Open\\
EV20811 & 132 & DO & Discharge BV F.O. & D12516 & Open & Close\\
ZS20854 & 109 & DI & Pump is on & MP-516T & Off & On\\
ZS20809 & 130 & DI & Pump is on & MP-516A & Off & On\\
ZSC20850 & 112 & DI & TT to D-301 BV ZSC & D12516 & close & Open\\
ZSC20851 & 113 & DI & discharge BV ZSC & D12516 & close & Open\\
ZSC20816 & 131 & DI & Outlet N2 clear BV ZSC & D12516 & close & Open\\
ZSC20815 & 133 & DI & Outlet BV ZSC & D12516 & close & Open\\
ZSC20811 & 132 & DI & Discharge BV ZSC & D12516 & Closed & Open\\
ZSO20811 & 149 & DI & discharge BV ZSO & D12516 & Open & Closed\\
CV20812 & 104 & AO & E-12516C Cooler CV F.C & D12516 & 0 & 110\\
CV20813 & 105 & AO & E-12516C Heater CV F.O. & D12516 & 0 & 110\\
CV20814 & 102 & AO & E-12516C Steam CV F.C & D12516 & 0 & 110\\
EV40215 & 428 & DO & NC-701 BV & D301 & Close & Open\\
ZSC40215 & 429 & DI & NC-701 BV ZSC & D301 &  & \\
LIT20805 & 140 & AI & dP Level [\%] & D12516 & 0 & 100\\
LSH20802 & 183 & AI & Level Switch & D12516 & high & not high\\
\end{tabular}
\end{center}

Based on the proposed control narrative, the following IO count would be required to realize the project. 
\begin{center}
\begin{tabular}{ccc}
S.No. & IO Type & IO Qty\\
\hline
1 & AI & 11\\
2 & AO & 3\\
3 & DI & 9\\
4 & DO & 8\\
\end{tabular}
\end{center}

\clearpage
\section{Implementation Strategy}
\label{sec:orgfb8ca20}
\begin{figure}[htbp]
\centering
\includegraphics[width=15cm,height=12cm]{/home/prasanna/Documents/git/CSR/images/architecture.jpg}
\caption{Implementation architecture showing Arduino, SQL dB \& Web-browser interface}
\end{figure}
\clearpage
\subsection{Hardware}
\label{sec:org978ce6f}
Based on the above architecture, Arduino Mega 2560 shall be used as
the microcontroller for interfacing all the field IO. The AO shall be
converted or treated as Digital output as there are no analog output
pins on such microcontroller family. The temperature control loop can
be simulated for the purpose of demonstration. 

The Mega2560 can be interfaced to an Ethernet Shield, by stacking the
Ethernet Shield over the arduino board. This enables to aruduino to
communicate to the PC over WiFi. This communication will be used to
receive process values from the controller to the PC for logging and
display purposes.  A WiFi router will be used establish the local
network consisting of arduino (with WiFi shiled) and PC.

The PC is a general purpose laptop that will be used for programming
the arduino. The PC will also serve as an HMI to display the process
values and log the process values which will be used for trending.

\subsection{Software}
\label{sec:org1cc2662}
The aruduino will be configured as a webserver and the PC will act as
a client. The following sofware will be used for this implementation:
\begin{itemize}
\item Arduino programmer \emph{\emph{check}}
\item WAMP / LAMP stack
\begin{itemize}
\item Windows / Linux
\item Apache web server
\item mySQL (database server)
\item PHP or Python
\item HTML/CSS/JavaScript
\end{itemize}
\end{itemize}

\subsubsection{Arduino programmer \emph{\emph{check}}}
\label{sec:org12b928f}
This is used to develop debug and download the code to the
controller. The control files are created with .ino extension

\subsubsection{WAMP/LAMP stack}
\label{sec:org1904666}
\begin{enumerate}
\item Windows/Linux
\label{sec:orga807008}
The operating system for the PC could either be Windows / Linux. This
will be the host for the other software such as the webserver,
database server and Python/PHP.
\item Apache Webserver
\label{sec:orgd2b9e2b}
<< \emph{Insert Apache configuration here} >>
\item mySQL
\label{sec:orgc3fa1a0}
<< \emph{Insert mySQL configuration here} >>
\item PHP/Python
\label{sec:orgacaee96}
<< \emph{Insert PHP/Python code here} >> 

PHP code identified for use is listed below. The following code is a
placeholder and needs to be further customized to have a complete
working code.

\begin{itemize}
\item dbconfig.php
\label{sec:orgc6a7b30}
\begin{verbatim}
<?php
define('DB_USER', "testuser");     // Your database user name
define('DB_PASSWORD', "test123");  // Your database password (mention your db password here)
define('DB_DATABASE', "TESTDB");   // Your database name
define('DB_SERVER', "localhost");  // db server (Mostly will be 'local' host)

?>
\end{verbatim}

\item dbconnect.php
\label{sec:orge46cf52}
\begin{verbatim}
<?php

class DB_CONNECT {

    // Constructor
    function __construct() {
	// Trying to connect to the database
	$this->connect();
    }

    // Destructor
    function __destruct() {
	// Closing the connection to database
	$this->close();
    }

   // Function to connect to the database
    function connect() {

	//importing dbconfig.php file which contains database credentials 
	$filepath = realpath (dirname(__FILE__));

	require_once($filepath."/dbconfig.php");

		// Connecting to mysql (phpmyadmin) database
	$con = mysql_connect(DB_SERVER, DB_USER, DB_PASSWORD) or die(mysql_error());

	// Selecing database
	$db = mysql_select_db(DB_DATABASE) or die(mysql_error()) or die(mysql_error());

	// returing connection cursor
	return $con;
    }

	// Function to close the database
    function close() {
	// Closing data base connection
	mysql_close();
    }

}

?>
\end{verbatim}
\item insert.php
\label{sec:orgef94d73}
\begin{verbatim}
<?php

header("Access-Control-Allow-Origin: *");
header("Content-Type: application/json; charset=UTF-8");

//Creating Array for JSON response
$response = array();

// Check if we got the field from the user
if (isset($_GET['temp']) && isset($_GET['hum'])) {

    $temp = $_GET['temp'];
    $hum = $_GET['hum'];

    // Include data base connect class
    $filepath = realpath (dirname(__FILE__));
	require_once($filepath."/db_connect.php");


    // Connecting to database 
    $db = new DB_CONNECT();

    // Fire SQL query to insert data in weather
    $result = mysql_query("INSERT INTO weather(temp,hum) VALUES('$temp','$hum')");

    // Check for succesfull execution of query
    if ($result) {
	// successfully inserted 
	$response["success"] = 1;
	$response["message"] = "Weather successfully created.";

	// Show JSON response
	echo json_encode($response);
    } else {
	// Failed to insert data in database
	$response["success"] = 0;
	$response["message"] = "Something has been wrong";

	// Show JSON response
	echo json_encode($response);
    }
} else {
    // If required parameter is missing
    $response["success"] = 0;
    $response["message"] = "Parameter(s) are missing. Please check the request";

    // Show JSON response
    echo json_encode($response);
}
?>
\end{verbatim}
\item readall.php
\label{sec:org33845a3}
\begin{verbatim}
<?php

header("Access-Control-Allow-Origin: *");
header("Content-Type: application/json; charset=UTF-8");


//Creating Array for JSON response
$response = array();

// Include data base connect class
$filepath = realpath (dirname(__FILE__));
require_once($filepath."/db_connect.php");

 // Connecting to database 
$db = new DB_CONNECT();	

 // Fire SQL query to get all data from weather
$result = mysql_query("SELECT *FROM weather") or die(mysql_error());

// Check for succesfull execution of query and no results found
if (mysql_num_rows($result) > 0) {

	// Storing the returned array in response
    $response["weather"] = array();

	// While loop to store all the returned response in variable
    while ($row = mysql_fetch_array($result)) {
	// temperoary user array
	$weather = array();
	$weather["id"] = $row["id"];
	$weather["temp"] = $row["temp"];
		$weather["hum"] = $row["hum"];

		// Push all the items 
	array_push($response["weather"], $weather);
    }
    // On success
    $response["success"] = 1;

    // Show JSON response
    echo json_encode($response);
}	
else 
{
    // If no data is found
	$response["success"] = 0;
    $response["message"] = "No data on weather found";

    // Show JSON response
    echo json_encode($response);
}
?>
\end{verbatim}
\end{itemize}
\item HTML/CSS/JavaScript
\label{sec:orgdc0e42c}
<< \emph{Insert HTML/CSS/JavaScript code here} >>
\item {\bfseries\sffamily TODO} 
\label{sec:org3d54258}
\begin{itemize}
\item[{$\square$}] include org-babel execute function for HTML
\item[{$\square$}] create stylesheet and move CSS information
\item[{$\square$}] create object for valve - two part valve - left side shows DO - right side shows feedback
\end{itemize}
\item HTML code
\label{sec:org8de05df}
\begin{verbatim}
<!DOCTYPE html>
<html>
  <head>
    <link rel="stylesheet" type="text/css" href="css/index.css">
</head>
<body>
  <img src="img/D301.png">
<p id="tt853"> </p>
<p id="pt852"> </p>
<p id="tt807"></p>
<p id="pit804"></p>
<p id="lit805"></p>
<p id="tt806"></p>
<p id="tt808"></p>
<p id="pt810"></p>
<script type="text/javascript">
    document.getElementById('tt853').innerHTML = "35 degC";
    document.getElementById('pt852').innerHTML =  "100 psig";
    document.getElementById('tt807').innerHTML = "23 degC";
    document.getElementById('pit804').innerHTML = "6 psig";
    document.getElementById('lit805').innerHTML = "12 %";
    document.getElementById('tt806').innerHTML = "15 degC";
    document.getElementById('tt808').innerHTML = "17 degC";
    document.getElementById('pt810').innerHTML = "20 psig";    
</script>

</body>
</html>
\end{verbatim}

\item CSS code
\label{sec:orgc4a38b4}
\begin{verbatim}

#tt853   {
	position: absolute;
	left: 350px;
	top: 110px;
	color: blue;
	font-weight: bold;
}
#pt852   {
	position: absolute;
	left: 525px;
	top: 110px;
	color: blue;
	font-weight: bold;
}
#tt807   {
	position: absolute;
	left: 340px;
	top: 290px;
	color: blue;
	font-weight: bold;
}
#pit804   {
	position: absolute;
	left: 170px;
	top: 510px;
	color: blue;
	font-weight: bold;
}
#lit805   {
	position: absolute;
	left: 260px;
	top: 595px;
	color: blue;
	font-weight: bold;
}
#tt806   {
	position: absolute;
	left: 189px;
	top: 685px;
	color: blue;
	font-weight: bold;
}
#tt808   {
	position: absolute;
	left: 410px;
	top: 670px;
	color: blue;
	font-weight: bold;
}
#pt810   {
	position: absolute;
	left: 610px;
	top: 670px;
	color: blue;
	font-weight: bold;
}
\end{verbatim}
\end{enumerate}

\section{Bill of Material}
\label{sec:orge28e68a}
<< \emph{Insert BOM here} >>
\clearpage
\section{Conclusion}
\label{sec:orgc24ae22}
The proposed solution in this paper shall meet its objective of
reducing the cycle time of D-301 blender by removing the dependency of
the truck for NC-701 material. It will also ensure that the quality of
the blending product is improved by regulating the product
temperature.
\section{References}
\label{sec:org70fc1e5}
\begin{enumerate}
\item \href{https://www.controlglobal.com/assets/knowledge\_centers/abb/assets/Benefits-of-state-based-control-white-paper.pdf}{Benefits of State Based Control}
\end{enumerate}
\end{document}
